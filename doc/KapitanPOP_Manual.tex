% Comments start with a %, and are ignored by LaTeX.

% In LaTeX, every document starts with a documentclass.  I strongly
% recommend "memoir" over any of the more standard ones.
\documentclass[article,12pt]{memoir}

% This part of the document is called the preamble.

% Command to use times new roman.  I am not a fan of computer modern,
% the default font.
\usepackage{mathptmx}

% Example numbering
\usepackage{gb4e}

\usepackage{natbib}
\bibpunct{(}{)}{;}{a}{}{,}

% Math mode
\usepackage{amsmath}

\begin{document}
	% Now the body of the document begins.
	
	\bibliographystyle{linquiry2}
	
	\title{KapitanPOP Manual}
	\author{Oleksandr Rudyy}
	
	\maketitle
	\chapter{Metrics Computation}
	This section accurately describes how KapitanPOP computes its metrics from the data contained in the HDF5 trace file. First, it starts with a depiction of how raw metrics are extracted, i.e. \textit{Useful Computation Time}, evolving gradually to the higher level metrics, like \textit{Parallel Efficiency}. Before continuing, though, it is convenient to give a solid definition to some very used terms in order to avoid confusions.
	
	\begin{itemize}
		\item \textbf{Useful}: this adjective refers to any value belonging to the \textit{Running} state defined by Extrae. In other words, \textit{useful} is anything that belongs to user code, or in other words, that is not part of system's -the kernel- or runtime's -OpenMP, pthreads, MPI- code. Input/Output or I/O can also be considered as \textit{useful}, this will depend of how Extrae was configured when tracing.
		\item \textbf{Maximum}: refers to the maximum existing value when grouping the data by application, process and thread. For example, \textit{Maximum Time in MPI} is the maximum time one process -MPI rank- has spent inside MPI compared to the rest of processes.
		\item \textbf{Runtime}: is the elapsed time since the start of an application to its end as recorded by Extrae.
		\item \textbf{Ideal}: this adjective refers to any value that has been computed from a Dimemas trace with ideal network -zero latencies, infinite bandwidth.
		\item \textbf{Reference}: is used for scalability purposes and refers to the metric coming from the run with less number of cores. 
	\end{itemize}
	\subsection{Raw Metrics}
	\begin{itemize}
		\item Runtime
		\item Useful Compute Time
		\begin{itemize}
			\item Total Useful Compute Time
			\item Maximim Useful Compute Time
			\item Average Useful Compute Time
		\end{itemize}
		\item Total Useful Instructions
		\item Total Useful Cycles
		\item Ideal Times
		\begin{itemize}
			\item Ideal Runtime
			\item Maximum Ideal Useful Compute Time
		\end{itemize}
	\end{itemize}
	
	\subsection{Runtime}
	The \textit{runtime} is extracted from Paraver's header, thus it is already provided and not computed.
	\subsubsection{Useful Compute Time}
	The Useful Compute Time is the time the application spends running user code. It is computed from \textit{States} records where state is equal to \textit{Running}.
	\begin{equation}\label{totustime}
		\begin{gathered}
			\text{Total Useful Compute Time} =\\
			\sum StateRecord_{time\_end} - \sum StateRecord_{time\_start}\\
			\text{where~$StateRecord_{state} =$ Running}
		\end{gathered}
	\end{equation}
	
	\begin{equation}\label{maxustime}
		\begin{gathered}
			\text{Max. Useful Compute Time} =\\
			\max(\sum StateRecord_{time\_end} - \sum StateRecord_{time\_start}),\\
			\text{where~$StateRecord_{state} =$ Running, groupby(appID, taskID, theadID)}
		\end{gathered}
	\end{equation}

	\begin{equation}\label{avustime}
		\begin{gathered}
			\text{Average Useful Compute Time} =\\
			\text{Total Useful Compute Time } / \text{ number of threads}
		\end{gathered}
	\end{equation}
	
	\subsubsection{Total Useful Instructions}
	\begin{equation}\label{totusins}
		\begin{gathered}
			\text{Total Useful Instructions} =\\
			\sum EventRecord_{EvValue},\\
			\big(\text{where~$EventRecord_{EvType} = PAPI\_TOT\_INS$ }\wedge\\
			EventRecord_{time} = StateRecord_{time\_end}, \text{where~$StateRecord_{state} =$ Running}\big)\\
			\text{groupby(appID, taskID, theadID)}
		\end{gathered}
	\end{equation}
	
	\subsubsection{Total Useful Cycles}
	\begin{equation}\label{totuscyc}
	\begin{gathered}
	\text{Total Useful Cycles} =\\
		\sum EventRecord_{EvValue},\\
		\big(\text{where~$EventRecord_{EvType} = PAPI\_TOT\_CYC$ }\wedge\\
		EventRecord_{time} = StateRecord_{time\_end}, \text{where~$StateRecord_{state} =$ Running}\big)\\
		\text{groupby(appID, taskID, theadID)}
	\end{gathered}
	\end{equation}
	
	\subsubsection{Ideal Times}
	\textit{Ideal Runtime} and \textit{Maximum Ideal Useful Compute Time} are computed the same way as \textit{Runtime} and \textit{Maximum Useful Compute Time} with the difference that the data used comes from a Dimemas trace simulating an ideal network.
	
	\section{Multiplicative Performance Metrics}
	\begin{itemize}
		\item Parallel Performance Metrics
		\begin{itemize}
			\item Load Balance
			\item Communication Efficiency
			\item Transfer Efficiency
			\item Serializtion Efficiency
			\item Parallel Efficiency
		\end{itemize}
		\item Global Efficiency
		\item Scalabilities
		\begin{itemize}
			\item IPC Scalability
			\item Frequency Scalability (GHz)
			\item Strong
			\begin{itemize}
				\item Speedup
				\item Computation Scalability
				\item Instruction Scalability
			\end{itemize}
			\item Weak
			\begin{itemize}
				\item Load Increase Factor (LIF)
				\item Speedup
				\item Computation Scalability
				\item Instruction Scalability
			\end{itemize}
		\end{itemize}
		\item Serial Performance Metrics
		\begin{itemize}
			\item Average IPC
			\item Average Frequency
		\end{itemize}
	\end{itemize}
	
	\subsection{Parallel Performance Metrics}
	
	\begin{equation}\label{loadbalance}
		\begin{gathered}
			\text{Load Balance} =\\
			\text{Average Useful Compute Time } / \text{ Max. Useful Compute Time} * 100
		\end{gathered}
	\end{equation}
	
	\begin{equation}\label{commeff}
		\begin{gathered}
			\text{Communication Efficiency} =\\
			\text{Max. Useful Compute Time } / \text{ Runtime} * 100
		\end{gathered}
	\end{equation}
	
	\begin{equation}\label{transfeff}
		\begin{gathered}
			\text{Transfer Efficiency} =\\
			\text{Ideal Runtime } / \text{ Runtime} * 100
		\end{gathered}
	\end{equation}
	
	\begin{equation}\label{serialeff}
		\begin{gathered}
			\text{Serialization Efficiency} =\\
			\text{Communication Efficiency } / \text{ Transfer Efficiency} * 100
		\end{gathered}
	\end{equation}
	
	\begin{equation}\label{pareff}
		\begin{gathered}
			\text{Parallel Efficiency} =\\
			\text{Load Balance} * \text{Communication Efficiency} / 100
		\end{gathered}
	\end{equation}
	
	\subsection{Global Efficiency}
	
	\begin{equation}\label{globeff}
		\begin{gathered}
			\text{Global Efficiency} =\\
			\text{Parallel Efficiency} * \text{Computation Scalability} / 100
		\end{gathered}
	\end{equation}
	
	\subsection{Scalabilities}	
	
	\begin{equation}\label{ipcscala}
		\begin{gathered}
			\text{IPC Scalability} =\\
			\text{Average IPC } / \text{ Reference Average IPC} * 100
		\end{gathered}
	\end{equation}
	
	\begin{equation}\label{freqscala}
		\begin{gathered}
			\text{Frequency Scalability (GHz)} =\\
			\text{Average Frequency } / \text{ Reference Average Frequency} * 100
		\end{gathered}
	\end{equation}
	
	\subsubsection{Strong}
	
	\begin{equation}\label{strongspeedup}
		\begin{gathered}
			\text{Speedup} =
			\text{Reference Runtime } / \text{ Runtime}
		\end{gathered}
	\end{equation}
	
	\begin{equation}\label{strongcompscala}
		\begin{gathered}
			\text{Strong Computation Scalability} =\\
			\text{Reference Total Useful Compute Time} / \text{Total Useful Compute Time} * 100
		\end{gathered}
	\end{equation}
	
	\begin{equation}\label{stronginscala}
		\begin{gathered}
			\text{Strong Instruction Scalability} =\\
			\text{Reference Total Useful Instructions} / \text{ Total Useful Instructions} * 100
		\end{gathered}
	\end{equation}
	
	\subsubsection{Weak}
	\textbf{LIF} (Load Increase Factor) is the factor by which the load (amount of work) increases. Ideally, it should be equal to the increase in number of processes/threads.
	\begin{equation}\label{lif}
		\begin{gathered}
			\text{LIF} =\\
			\text{Number of threads } / \text{ Reference Number of Threads}
		\end{gathered}
	\end{equation}
	
	\begin{equation}\label{strongspeedup}
		\begin{gathered}
			\text{Speedup} =
			\text{Reference Runtime } / \text{ Runtime} * LIF
		\end{gathered}
	\end{equation}
	
	\begin{equation}\label{weakcompscala}
		\begin{gathered}
			\text{Weak Computation Scalability} =\\
			\text{Reference Total Useful Compute Time } / \text{ Total Useful Compute Time} * LIF * 100
		\end{gathered}
	\end{equation}
	
	\begin{equation}\label{weakinscala}
		\begin{gathered}
			\text{Weak Instruction Scalability} =\\
			\text{Reference Total Useful Instructions } / \text{ Total Useful Instructions} * LIF * 100
		\end{gathered}
	\end{equation}
	
	\subsection{Serial Performance Metrics}
	
	\begin{equation}\label{avipc}
		\begin{gathered}
			\text{Average IPC} =\\
			\text{Total Useful Instructions } / \text{ Total Useful Cycles}
		\end{gathered}
	\end{equation}
	
	\begin{equation}\label{avipc}
		\begin{gathered}
			\text{Average Frequency (GHz)} =\\
			\text{Total Useful Cycles } / \text{ Total Useful Time (ms)} / 1000
		\end{gathered}
	\end{equation}
	
\end{document}


